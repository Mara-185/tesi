% --------------------------------------------
%		ABSTRACT
%---------------------------------------------

\documentclass[10pt,a4paper,twoside]{report}
\usepackage[T1]{fontenc}
\usepackage[italian,english]{babel}
\usepackage{hyperref}
\addto\extrasenglish{%
	\def\chapterautorefname{Chapter}}
\addto\extrasenglish{%
	\def\subsectionautorefname{section}}
\usepackage{graphicx}
\usepackage{rotating}
\usepackage{tabularx}
\usepackage{amsmath}
\usepackage{listings}
\usepackage[printonlyused,withpage]{acronym}
\usepackage{frontespizio}
\usepackage{imakeidx}
\usepackage{comment}
\usepackage[table,usenames,dvipsnames]{xcolor}
\usepackage{xcolor}
\usepackage{setspace}
\usepackage{titlesec}
\usepackage{chemformula}
\usepackage{varioref}
\usepackage{subfigure}
\usepackage{alphalph}
\usepackage{siunitx}
\usepackage{multirow}
\usepackage{array}
\usepackage[nottoc]{tocbibind}
\usepackage[a4paper]{geometry}
\usepackage{pdfpages}


%\usepackage{url}
%\usepackage{listings}
%\usepackage[toc]{appendix}

%[display]

\titleformat{\chapter}%
  {\normalfont\bfseries\huge}{\thechapter.}{10pt}{}

% Cartelle da cui recuperare le immagini
\graphicspath{{immagini/}, {grafici/}, {first/}, {second/}, {third/},{fourth/}, {fifth/}}

%Indice
%\makeindex

%Indice dei contenuti
\hypersetup{
    colorlinks = false,
    allbordercolors = {white}
}

%Numerazione subsubsection
\setcounter{tocdepth}{3}
\setcounter{secnumdepth}{3}


%COMMAND TO INCREASE POSSIBLE LABELS IN CAPTION
\renewcommand*{\thesubfigure}{(\alphalph{\value{subfigure}})}

%COMMAND TO DEFINE NEW TYPE OF COLUMN(per andare a capo e centrare il testo)
\newcolumntype{C}[1]{>{\centering\let\newline\\\arraybackslash\hspace{0pt}}m{#1}}

%VARIOREF
\vrefwarning





\begin{document}

\begin{center} \bfseries
\LARGE Study of monolithic CMOS pixel sensors in the Belle II experiment upgrade\\

\vspace{5mm}
\textsc{\large abstract}
\end{center}
%\chapter*{\centering \Large Study of monolithic CMOS pixel sensors in the Belle II experiment upgrade}

Belle II is a particle physics experiment located at the KEK laboratory in Tsukuba (Japan). The detector is a general-purpose spectrometer to study electron-positron collisions produced by the SuperKEKB accelerator, a second generation flavor-factory which operates at the luminosity frontier, holding the world record of instantaneous luminosity with $L_{peak}$ = \num{4.7e34} $cm^{-2} s^{-1}$. 

SuperKEKB is the upgrade of the preceding facility KEKB (operational from 1998 to 2016) and it consists in a 3 km-circumference asymmetric accelerator which collides electrons and positrons beams at a center-of-mass energy near the $\Upsilon$(4S) resonance ($\sqrt{s}$ = 10.58 GeV). It started its data taking in March 2019.

In the next decade, the collider aims to collect  an unrivaled dataset of 50 $ab^{-1}$ (x50 Belle dataset, x100 BaBar dataset) and to reach a peak luminosity of \num{6e35} $cm^{-2} s^{-1}$. This will allow to study the charge-parity violation in B mesons system with more precision and to search for new hints of pyhsics beyond the Standard Model.

To achieve these challenging targets, it will be necessary a significant upgrade of the accelerator and its main components (like the injection system and the equipment nearby the interaction region), probably requiring the installation of a new detector. As a matter of fact, to the increase in luminosity corresponds not only large data collected and greater possibility to study rare processes, but also higher doses of radiation and larger backgrounds, which could undermine the integrity and the operation of the Belle II detector.
In particular the subdetectors which are closest to the beam pipe are those more exposed to severe conditions, like the vertex detector (VXD), composed of the inner pixel detector (PXD, made of layers of pixels) and the outermost silicon vertex detector (SVD, made of layers of strips). They allow the reconstruction of charged particle tracks and of decay vertices with high performance. Recent studies have shown that the current detector could operate efficiently up to a luminosity of $L_{ist}$ = \num{2e35} $cm^{-2} s^{-1}$, but safety margins are not so large. 
Consequently, in this context, different upgrade projects have been proposed, which intend to design a new vertex detector, making it more resistant even in harsher working conditions, while the luminosity will be gradually increased. 

This work focuses especially on the VerTeX Detector (VTX) proposal (the one chosen for the final upgrade), whose program provides for replacing the whole VXD with fully-pixelated five layers at different distances from the beam pipe, equipped with the same type of sensor technology, which is the CMOS Depleted Monolihic Active Pixel Sensors (DMAPS). 

The good results achieved by the ALICE experiment (LHC, CERN), which employed the same technology, have suggested this solution which has proven to be reliable and promising in maintaining low occupancy, despite the worse expected background environment, and good radiation hardness even after irradiation. 

In order to fullfil the physics requirements of Belle II experiment, a new silicon sensor is being designed, called OBELIX, exploiting the 180 nm TowerJazz Semiconductor process. Developments will ensure a faster, lighter and highly granular chip, reducing the material budget and improving tracks and vertices reconstruction. 

OBELIX planning is based on studies done on previous prototypes, among which TJ-Monopix 2, whose characterization is the main topic of this thesis. Laboratory and beam tests have been conducted and are still in progress, in order to study the efficiency of the chip before and after irradiation, its power consumption, and to fully characterize its electrical characteristics. 
In particular, we have characterized the response of the pixel matrix, extracting important results that have allowed to interpret data taken during the Test Beam at Desy (July 2022), and that are being used in the design of the OBELIX chip. 
In more details, the threshold distributions for all the different types of front-end circuits implemented in the matrix have been studied, together with their dispersion and noise distributions. 
The calibration of the Time Over Threshold curves (which is a time width signal processing method used in this prototype) have been done by internal injection tests. The absolute calibration of the whole matrix have been achieved, employing a \ch{^{55}Fe} radioactive source. Other radioactive sources have been used too, in order to check the trend of the ToT curves for charge values not accessible by internal injection. 
Additionally, different register settings have been examined in the interest of operating the matrix at low threshold, that is crucial to keep high efficiency even after irradiation. For this reason, several tests have been conducted to tune the threshold, in order to reduce the dispersion and make the threshold on the matrix as uniform as possible.
During this investigation, a cross-talk issue has been discovered and therefore studied to understand its causes and possible solutions to mitigate this effect.


\end{document}