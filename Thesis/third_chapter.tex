% --------------------------------------------
%		CAPITOLO 3 
%---------------------------------------------
\chapter{VTX detector}

This chapter focuses on one of the four proposal for the vertex detector upgrade of Belle II, that is VTX. After a brief reference to the reasons behind the vertex detector upgrade, we will go trough VTX concept and layout, designed with a new geometry with respect VXD and so with a different mechanical structure and a new pixel sensor, in order to fullfil the new requirements dictated by new environment conditions. Moreover all ongoing studies are supported by continual tests and simulations that we will also take a look at.


%---------------------------------------------
%			3.1
%---------------------------------------------
\section{VTX Layout}

In section \vpageref{nano_beam} we have introduced in a few word the concept of the \textit{nano-beam} scheme, which could allow to achieve the new fixed target of istantaneous luminosity. This new strategy required a strong focusing of the beams in particular at the IP, resulting in a large amounts of beam induced background and as consequence in a higher dose of radiation in the innermost detector layers. 

So VTX aims to replace the all VXD with a fully pixelated detector based on Depleted Monolithic Active Pixel Sensors (DMPAS) arranged on five layers at different distance from the beam pipe. As already discussed for other upgrade proposal, it may be important to try to reduce the material budget, in order to minimize the multiple Coulomb scattering which particularly affects the very soft particles produced in Belle II collisions.

\subsection{iVTX}

\subsection{oVTX}


\begin{comment}
%---------------------------------------------
%			3.2
%---------------------------------------------
\section{Performance simulation}



%---------------------------------------------
%			3.3
%---------------------------------------------
\section{Obelix chip design}


%% OBELIX


%---------------------------------------------
%			3.4
%---------------------------------------------
\section{Mechanical structure}


%% Struttura Layer
%% Sistema di cooling 
%% Connessioni?

\end{comment}




%---------------------------------------------
%		BIBLIOGRAFIA
%---------------------------------------------




%---------------------------------------------
%		COMMENT
%---------------------------------------------
