% --------------------------------------------
%		CAPITOLO 3 
%---------------------------------------------
\chapter{VTX detector}

%REQUIRED RESOLUTION FROM CDR 15UM


This chapter focuses on one of the four proposal for the vertex detector upgrade of Belle II, that is VTX. After a brief reference to the reasons behind the vertex detector upgrade, we will go trough VTX concept and layout, designed with a new geometry with respect VXD and so with a different mechanical structure and a new pixel sensor, in order to fullfil the new requirements dictated by new environment conditions. Moreover all ongoing studies are supported by continual tests and simulations that we will also take a look at.


%---------------------------------------------
%			3.1
%---------------------------------------------
\section{VTX Layout and mechanical structure?}

In section \vpageref{nano_beam} we have introduced in a few word the concept of the \textit{nano-beam} scheme, which could allow to achieve the new fixed target of istantaneous luminosity. This new strategy required a strong focusing of the beams in particular at the IP, resulting in a large amounts of beam induced background and as consequence in a higher dose of radiation in the innermost detector layers, which therefore have to be robust enough to keep good performance.
Furthermore, to be able to reach the target luminosity, SuperKEKB might have to consider an improvements of the final focusing magnets and so a potentially re-design of the interaction region, including the detector (regardless of the hit rates and radiation hardness issue?).\\

\begin{figure}[h!]
\centering
\includegraphics[scale=.6]{VTX_layers}
\caption{Concept of VTX layout with 5 barrel layers, filling the current VXD volume.}
\label{fig:VTX_layers}
\end{figure}


So VTX aims to replace the all VXD with a fully pixelated detector based on Depleted Monolithic Active Pixel Sensors (DMPAS) arranged on five layers at different distance from the beam pipe. Actually the radii and the number of the layers are subject to studies and simulations at the moment, in order to achieve an optimized arrangement(?) (figure \vpageref{fig:VTX_layers}).  For now two layers are planned in the innermost part (\textit{i}VTX) and three in the outermost (\textit{o}VTX). The active lenght of the ladders is expected to vary from 12 to 70 cm to cover the required acceptance of $17^{\circ} < \theta < 150^{\circ}$.
As already discussed for other upgrade proposal, it may be important to try to reduce the material budget, in order to minimize the multiple Coulomb scattering which particularly affects the very soft particles produced in Belle II collisions. By using a single sensor type, it is expected a reduction of the overall material budget up to 2\% of radiation lenght, against the present 3\% of VXD, which uses two different sensors such as pixels and strips.


\subsection{iVTX}

The \textit{internal}VTX consists of the first two detector layers devised togheter with a self-supported air-cooled all-silicon ladder concept, where four contiguous sensors are diced out of a wafer, thinned and interconnected with post-processed redistribution layers. They are designed to be at 1.4 and 2.2 cm respectively from the beam pipe, and target an individual material budget of about 0.1\% radiation lenght. This is actually achivable because the overall surface of this layers is moderate, below 400 $cm^{2}$, the low sensor power dissipation and the few connections needed for the operation. Precisely for these reasons, air cooling could be a workable system to avoid overheating.


\begin{figure}[h!]
\centering
\includegraphics[scale=.65]{iVTX}
\caption{Sketch of the all-silicon ladder concept of the iVTX. Four dummy sensors are shown in blue on the silicon support in grey. The yellowish lines instead, indicate power and data transmission lines (RDL). Power is delivered to the ladder by a flex cable, which also transmits data to and from the chips in the final chips.}
\label{fig:iVTX}
\end{figure}

The ladder has to be equipped with four Obelix chips and thinned to 50 $\mu$m except in some border regions, where a few hundreds of $\mu$m are necessary to ensure mechanical stability. 
In order to interconnect the sensors along the ladder and provide a unique connector at the backward end, during the post-processing metal strips are etched on the redistribution layer (RDL). The latter has the main purpose to route power and data via impedence-controlled transmission lines to a flex cable, added at the end of the ladder.
After the RDL processing, the backside of the ladder has to be thinned in accordance with what was previously mentioned.


In figure \vpageref{fig:iVTX} is showing a sketch of the iVTX demonstrator ladder, 140 mm long and 22 mm wide (grey). Instead of the actual sensors, it is equipped with four dummies chips with a lenght of about 30 mm (blue), which are used as resistor to mimic the estimated heat load in order to test the air cooling system and more generally to characterized the electrical, mechanical and thermal performance of the ladder.
A redistribution layers (RDL) for power and data is also added to the demonstrator, to connect the chips with a flex cable at the end of the ladder (yellowish lines). In addition the wafer is thinned to 400 $\mu$m and the sensitive areas down to 40 $\mu$m, to test the mechanical integrity.


\subsubsection{R\&D}

The R\&D is ongoing and the full-silicon ladder concept is currently being assessed with industrial partners. First thinned ladders have been produced and characterised with different thickness and geometry, revealing a homogeneous thickness over an area of 10 $cm^{2}$. 

In addition tests are focused on evaluating power delivery efficiency, the quality of the signal which travel through the ladder and also the process used to fully assembly it. 
In figure \vpageref{fig:iVTX_eye} are shown eye diagrams from simulation with a transfer rate of 640 Mbps, which may imply that 320 Mbps of data rate will be possible. (??) 

\begin{figure}[h!]
\centering
\includegraphics[scale=.7]{iVTX_eye}
\caption{Eye diagrams of the iVTX data transmission lines at four different locations on the ladder.}
\label{fig:iVTX_eye}
\end{figure}

Moreover it has been demonstrated that air at $15^{\circ}$ flowing with a speed of 10 m/s succeeds to cool a single inner module, assuming power is uniformly dissipated on the sensor surface. The maximum temperature reached is $20^{\circ}$C. 

Through very first estimates it is expected that an equivalent section of 6 tubes with 10 mm of diameter is necessary to expel the heat from the inner layers, roughly equal to 65 W. So it's necessary to design a mechanical structure which forseens the space needed to the tubes in order to bring the air at the IP and also compatible with the new interaction region.


\subsection{oVTX}

The \textit{outer}VTX consists of three layers respectively at radii of 39 or 69, 89 and 140 mm from the beam pipe and because of the larger distance required to cover the acceptance, they are not self-support. They follow a more traditional approach, strongly inspired by the designed developed for the ALICE ITS2. Each ladder is water cooled and made of a light carbon fiber support structure, called \textit{truss}, which provide the mechanical integrity. Its structural design is showed in figure \vpageref{fig:oVTX}: 70 cm long and 5.8 g of weight, it is able to support more than 40 sensors in two rows next to each other with a small overlap, earning a material budget of 0.3\% $X_{0}$ for the first two layer and 0.8\%$X_{0}$  for the outermost one.

\begin{figure}[h!]
\centering
\includegraphics[scale=.7]{oVTX}
\caption{Prototype of the layer 5 \textit{truss}, which is the longest, made from thin carbon fibre structures.}
\label{fig:oVTX}
\end{figure}


For the cooling of the ladder, it is developing a cold-plate concept (figure \vpageref{oVTX_coldplate}), on which the sensors are glued and that in turn is installed on the truss. For each row, there is a polymide cooling tube that runs over all the sensors and turns back at the other end, so that the heated coolant leaves on the same side. Then two flex print cables to connect the two halves of the ladder to the connector.


\begin{figure}[h!]
\centering
\includegraphics[scale=.7]{coldplate}
\caption{A prototype of the cold-plate for colling. One coolant tube(golden) is connected to the cold plate(black) and turns 180° on the other end (not shown) so that the coolant flows in both directions and thus leaves on the same side it starts.}
\label{fig:oVTX_coldplate}
\end{figure}

For layer 3 instead, only one flex print cable in the backward side is considered in order to leave more space in the forward for other services and accelerator component. In addition for the third layer two different solutions are under study: at radius of 39 mm e 69 mm respectively. 
In figure \vpageref{fig:ladders} some example of possible solutions. 

\begin{figure}[h!]
\centering
\includegraphics[scale=.7]{ladders}
\caption{Schematic view of possibile solutions for the three outermost layers.}
\label{fig:ladders}
\end{figure}


In figure \vpageref{fig:oVTX_5} is shown the several structures described that shape a ladder of the outermost layer 5. From bottom to top come in succession the carbon fibre structure, two cold-plates for the two neighbouring sensor rows (Chips, in grey) and the flex cables for power and data transmission (green). 


\begin{figure}[h!]
\centering
\includegraphics[scale=.7]{oVTX_5}
\caption{An exploding drawing of a fully assembled layer 5 ladder.}
\label{fig:oVTX_5}
\end{figure}

\subsubsection{R\&D}

The carbon fiber support structure and flex cable have been designed and fabricated for the last ladder, which is also the longest. Also services for the last two ladders, like electrical connections and cooling, can be provided both on forward and backward sides.
A Multiline Power Bus has been realized in order to power each OBELIX chip along the ladder by a dedicated VDD and GND pair. \\

After the assembly described in the previous, first thermo-mechanical tests were performed and they show that the first resonance frequency is at 200 Hz, which is safely far from the one of the typical earthquakes in Japan and also that the thermal properties are good.\\

Trying to reduce as much as possible the material budget, the transmission lines and the flex cables has to be as thin as possible, but also need to ensure safe data transmission. Trace widths are trimmed to fulfill the same maximum voltage drop requirement (200 mV) for all the chips.

For this reason, the outermost ladders long 70 cm are equipped with two flex cables, one from each side of the \textit{truss}. In figure \vpageref{fig:oVTX_eye} the resulting eye diagram from testing the signal integrity of one of the 35 cm long transmission lines for data transmission rate of 500 Mbps. This result demonstrates that the bandwidth is large enough to allow more then needed 160 Mbps for data transmission.


\begin{figure}[h!]
\centering
\includegraphics[scale=.8]{oVTX_eye}
\caption{Eye diagram for the oVTX transmission line signal integrity of the layer 5 flex cable.}
\label{fig:oVTX_eye}
\end{figure}


In addition, thermal test have been performed for the last layer prototypes using kapton heaters to emulate the power dissipation of the chips. The coolant (demineralized water) has been set to a temperature of $10^{\circ}$ at the beginning, the environment at $20^{\circ}$ with a negative pressure 0f 0.2 bar. Results have been demonstrated that for three different configurations of the flow (such as monodirectional, bi-directional and with an U-turn at one end) the average temperature stands at $24^{\circ}$ with a maximum gradient of $\Delta T \approx 4^{\circ}$ along the full lenght of the ladder. 

All these test validate the design of the longest ladder, which is indeed the most challenging, and therefore the possibility to operate the chips safely.


\subsection{Thermomechanics and data transmission}

The proposed VTX detector intends to employ the same sensor type for all the layers in order to use a unique control and power supply system. It is expected to operate at room temperature and for what we have seen in the previous, the smaller cross section of data cables, the usage of optical fibers and the less complex cooling system might allow a considerable reduction of services with respect to the current VXD. This allows more room for maneuver in the design of the new IR, needed for ramping up the istantaneous luminosity in the future.

As consequence also the design of the mechanical support system, data cables and acquisition system required could be more simple and in particular, the standard PCIe40 acquisition boards used in Belle II match well the data throughput requirement.

%---------------------------------------------
%			3.2
%---------------------------------------------
\section{Performance simulation}
%richiesta sensitivity di 150 um ? VTX article, betagamma


%---------------------------------------------
%			3.3
%---------------------------------------------
\section{OBELIX chip design}

As we already seen, the VTX detector is designed with a single type sensor taylored to the specific need of Belle II, called OBELIX (Optimized BELle II pIXel sensor) and currently under development, based on fast and high granular Depleted Monolithic Active Pixel Sensor (DMAPS). This new sensor design comes from an evolution of TJ-Monopix 2, whose characterization is the main topic of this thesis, and which will be discussed in (reference) and both of them relie on the CIS? 180 $\mu$m process by TowerJazz Semiconductor.
As a matter of fact its predecessor is equipped with four different flavors (reference), and for now the final decision on which to use for Obelix has not been made. \\

\begin{figure}[h!]
\centering
\includegraphics[scale=.7]{obelix_schematic}
\caption{OBELIX chip design.}
\label{fig:obelix_scheme}
\end{figure}

A schematic layout of the chip is shown in figure \vpageref{fig:obelix_features}. The size of the sensor is expected to be 3 x 2 $cm^{2}$, with an active area of 3 x 1.5 $cm^{2}$ and an additional part in the periphery of about 3 x 0.3 $cm^{2}$ dedicated to data pre-processing and triggering. The pixel pitches are designed to be from 30 $\mu$m to 40 $\mu$m in both direction. 
To deal with the target hit rate of 120 MHz/$cm^{2}$, the timestamp clock signal can reach down to 25 ns, even if studies have demonstrated that a window of 100 ns is enough to limit to 320 Mbps the data throughput at the same expected hit rate.
With respect to TJ-Monopix 2, which is equipped with a triggerless column-drain readout without memory at the periphery, OBELIX must have a triggered readout architecture, in order to satisfy the needs of Belle II. Moreover the latency is fixed to 5 $\mu$s and it might operate up to 30 KHz trigger rate.
The expected power consumption instead, is expected to be about 200 mW/$cm^{2}$. 

Its main design features are summarised in \vpageref{fig:obelix_scheme}.


\begin{figure}[h!]
\centering
\includegraphics[scale=.7]{obelix_features}
\caption{Designed features of the OBELIX sensor.}
\label{fig:obelix_features}
\end{figure}





\begin{comment}
%---------------------------------------------
%			3.4
%---------------------------------------------
\section{Mechanical structure}

%% Struttura Layer
%% Sistema di cooling 
%% Connessioni?

\end{comment}




%---------------------------------------------
%		BIBLIOGRAFIA
%---------------------------------------------

%TVX ARTICLE
%CDR
%LUDO
%VTX proposal


%---------------------------------------------
%		COMMENT
%---------------------------------------------
