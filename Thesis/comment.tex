%-------------------------------------------------------------------------
%				COMMENT 1
%-------------------------------------------------------------------------

%1.1
\begin{comment}
Moreove new physics models searched in Belle II are those that include more specific flavor couplings, from which indirect researches can push the new physics scale much higher than the direct search programs.
\end{comment}

\begin{comment}
%già commentato
In general the detection of b $\rightarrow$ $\tau$ $\rightarrow$ \textit{l} transitions requires good lepton identification below 700 MeV/c (Low momentum track finding).
\end{comment}

%1.2
\begin{comment}
For these reasons, the supervision of the beams background becomes crucial both to reach the goal and to improve the possible precision measurements of physics. 

At present (currently) it is estimated that the background should remain accettable up to a luminosity value equal to \num{2.8e-35} $cm^{-2} s^{-1}$ with $\beta^{*}_{y}$= 0.6 mm.
As we shall see, the possibility(hope) to achieve higher luminosity is closely (strictly) relate to an upgrade plan of both the detector and the accelerator.
\end{comment}

\begin{comment}
[We can notice that the different balance between beam energies (lower beam energy in HER, higher in LER respect to KEKB) was chosen to reduce the beam losses due to Touschek scattering in the LER. This is expected to reduce the spatial separation between B mesons, studied in time-dependent CPV measurements, but leads to slight improvements in solid angle acceptance for missing energy decays].
\end{comment}

\begin{comment}
We can see briefly the most important parameters that influnce the mechanism?????

The overlap area of the beams is localized and its lenght is given by:

\begin{equation}
d = \frac{\sigma_{x}^{*}}{sin\phi_{x}}
\end{equation}

with $\phi_{x}$ defined as half of the crossing angle. The overlap lenght d represent the real lenght of the bunch to consider in the evaluation of the hourglass effect, and it is smaller than the effective bunch lenght along the direction of the beam axis. To reduce this effect, it's necessary to get:

\begin{equation}
\beta_{y}^{*} \geq d = \frac{\sigma_{x}^{*}}{sin\phi_{x}}
\end{equation}

Therefore to increase the $\beta$ function in the IP, the value of d has to diminish, decreasing as aconsequence the horizontal section in the IP and also increasing the crossing angle.
\end{comment}

%1.3

\begin{comment}
Belle II detector is a general-purpose spectrometers, optimized in precision measurements of B mesons and their decay products. Compared to its predecessor Belle, it have to preserve (mantain) good performance despite having smaller boost in the center of mass and suffering greater levels of backgrounds and so of radiations, which are one of the main causes of premature degradation of performance and expected mean life of the detector itself.\\

Belle II consists of a series of nested subdetectors, which surrounds the IP of the two beams, placed around the berillium beam pipe of 1 cm(10 mm??) of radius. Now we will go trough a briefly description of the several subdetectors. 
\end{comment}



\begin{comment}

\subsection{Alcune ulteriori modifiche rispetto a KEKB}

L'upgrade da KEKB a SuperKEKB, come in parte visto, ha richiesto importanti cambiamenti nei fasci e nello schema di collisione, tutto volto a raggiungere nuove vette di luimnosità. Altre fondamentali modifiche sono state fatte lungo l'anello di collisione, tra cui:

\begin{item}
\item rifacimento della regione d'interazione, che comprende 4 m intorno al PI, in modo da poter ospitare il nuovo detector Belle II, il sistema di focusing finale dei fasci e le due beam pipes;
\item il sistema a radiofrequenza è stato modificato per permettere una corrente maggiore dei fasci;
\item l'aggiunta di alcuni collimatori lungo entrambi gli anelli (11 nel LER  e 20 nel HER) per poter limitare il danno da radiazione sul rivelatore e i ''quenches'' dei magneti superconduttori, cioè un surriscaldamento dei suoi avvolgimenti che genericamente causa la perdità della superconduttività, dissipando la corrente circolante.
\item anche il sistema di vuoto è stato migliorato per limitare alcuni effetti collegati alla perdita di potenza del fascio, allungando quindi la vita media del fascio stesso.
\end{item}

\end{comment}


\begin{comment}

\subsection{Background monitoring system} [?]
%% Per le sorgenti di rumore si rimanda al capitolo 2 o lo metti nel capitolo 2?

Background is one of the most important problem for a particle detector, both for precision measurements of physics and for the performance of the different layer detector which constitute (make up) Belle II. For this reason several detectors are used to obtain measurements of radiation dose on both detector and delicate regions of the accelerator, to intervene as soon as possible in case of too high levels are reached. Indeed large doses of radiation could cause accidental damages on the detector, decreasing its performance.

In the chapter 2 (reference) we will go through the main (primary, leading) reasons of background. For now we mention some of the monitoring devices for the backgorund in SuperKEKB.


\begin{itemize}
\item Diamonds detector (called ''Diamonds') which control the rate of the radiation dose in the interaction region of the beam pipe. These also are part (take part) of the ''fast beam abort system'', which is a control system that collect (consider) data from different detector to evalute (valutare nel senso di decidere) the beam ''turning off'', in order to avoid that events out of control could cause damage (harm) to the whole structure.
\item CLAWS (sCintillation Light And Waveform Sensors), made by scintillator of plastic material and silicon photomultipliers, and used to monitor the Belle II background in the nearby (proximity) of the beam injection (main ring). Together with diamonds it takes part to the beam-abort system.
\item TPC's (Time Projection Chambers) which provides measurements (better) on the direction of the neutron flux in the tunnel which hosts (that houses) the accelerator.
\item $He^{3}$ tubes for the counting of thermal neutrons (which are those with a kinetic energy lower than 1/10 of eV, generally about 0.025 eV) around the Belle II detector.
\end{itemize}

\end{comment}