% --------------------------------------------
%		CHAPTER 1
%---------------------------------------------

\chapter{Belle II and SuperKEKB (SKB) accelerator}\label{ch:BelleII}

The first chapter introduces some of the main aspects of the Standard Model (SM), along with its open questions, on which the Belle II physics program is focused. A short description of the SuperKEKB accelerator and the Belle II detector's structure is also presented and in conclusion some highlights on the current state of measurements are shown.


%--------------------------------------------
%			    1.1
%--------------------------------------------
\section{Physics program of the B-factories}

The SM is a very successful theory describing three of the fundamental forces involving elementary particles: the strong, weak and electromagnetic interactions, with the exclusion of the gravitational force. It classifies all the elementary constituents of matter in 4 main groups, as shown in~\autoref{fig:sm}: quarks and leptons, constituting the matter fields; the gauge bosons, representing the interactions; and the Higgs boson, whose non-zero vacuum expectation value is needed to give mass to the otherwise massless matter fields.


\begin{figure}[h]
\centering
\includegraphics[scale=0.3]{SM}
\caption{Particle classification in the Standard Model.}
\label{fig:sm}
\end{figure}


%---------------------------------------------------------------------------------------


\subsection{Open questions in the Standard Model}


Despite the undeniable success of the SM in providing predictions for all known physics phenomena, which have been experimentally verified with high precision over the years, there are many fundamental aspects of nature on which it is unable to give answers \cite{physics_book}. Some of them are listed in the following.

\begin{itemize}
\item Three generations of quarks and leptons have been discovered, but it is not known whether they are the only ones and why.
\item The reasons behind the large differences in the masses of quarks and leptons (mass hierarchy) are unknown.
\item Although the Higgs mechanism is able to explain the cause of elementary particles' masses through spontaneous electro-weak symmetry breaking, it is not clear whether neutrinos can get their non-zero but very small masses through the interaction with the Higgs boson.
\item Another open question is the matter-antimatter asymmetry in the universe. Charge-Parity (CP) violation is necessary to explain the asymmetry, but the SM mechanism would predict a value several orders of magnitude smaller than what is needed to explain the matter domination over antimatter, which allowed the evolution of the universe as we know it today.
\item The flavour-changing currents are phenomenologically described with two mixing matrices: the Cabibbo-Kobayashi-Maskawa (CKM) matrix for the quarks and the Pontecorvo-Maki-Nakagawa-Sakata (PMNS) matrix for the neutrinos. The SM fails to provide any explanation neither for the structure of the matrices (only charged for the quarks, only neutral for the neutrinos), nor for the values of their elements (nearly diagonal for the quarks, highly non diagonal for the neutrinos).
\item Astrophysical observations indicate the existence of dark matter, but its origin and nature have not been explained yet.
\end{itemize}

All these open questions stimulate the search for new particles and processes that could provide more fundamental explanations.\\
At the energy frontier, experiments at the Large Hadron Collider (LHC) at CERN (CH) are looking for new particles created from the proton-proton collision with a center-of-mass energy up to \SI{14}{TeV}.\\
At the luminosity frontier, instead, new particles and interactions are searched through precision measurements of suppressed reactions or deviations from the SM. The discrepancies could be interpreted as a clue of new physics beyond the SM. The Belle II experiment at the SuperKEKB B-Factory is following this last approach.\\


In particular, the experiment investigates the CP violation in the B-mesons system and searches for new physics evidence in the decays of B and D mesons, in $\tau$ leptons and in the dark matter sector (DM).

%-----------------------------------------------------------------------------------------

\subsection{Peculiarity of asymmetric B factories} \label{sec:vertex_decay}

The SuperKEKB $e^{+}e^{-}$ collider operates with a center-of-mass energy of $\sqrt{s}$ = \SI{10.58}{GeV} at the $\Upsilon(4S)$ resonance, which decays almost instantaneously into a pair of B - anti B mesons in nearly 96\% of cases. 

In SuperKEKB the beam energies are different, leading to a center-of-mass boost in the laboratory that allows the reconstruction of the B-mesons decay vertices, their lifetimes, and the time-dependent decay rate asymmetries.

In a symmetric beams situation, the B mesons would be produced almost at rest, decaying roughly at the same point with undetectable decay length. 
The investigation of CP violating processes instead, requires measuring the decay time difference of the two B mesons and its uncertainty is dominated by that of the decay vertex measurement (order of hundreds microns), as we will now see in more detail.\\

SuperKEKB collides an electrons beam of \SI{7}{GeV} (High Energy Ring, HER) with a positrons beam of \SI{4}{GeV} (Low Energy Ring, LER) with a Lorentz boost factor of the $\Upsilon(4S)$ of $(\beta\gamma)_{\Upsilon(4S)}\approx0.28$.

The same boost is also acquired by the B-mesons, because they are produced almost at rest ($m_{\Upsilon(4S)}$ - $m_{2B_{0}}\approx$ \SI{19}{MeV}). Moreover knowing that $\tau_{B}\simeq $\SI{1.5e-12}{s} and so c$\tau_{B}\simeq$ \SI{450}{\micro m}, we can compute the average flight distance travelled before decaying:

\begin{equation}
\textit{l} = (\beta\gamma)_{\Upsilon(4S)}c\tau_{B} \approx \SI{126}{\micro m}  
\end{equation} 

This value must be within the vertex detector sensitivity in order to distinguish the vertex decay and as consequence to make precision measurements of lifetimes, mixing parameters and CP violation. The main task of the VerteX Detector (VXD) is to reconstruct the production and decay vertices of the particles originated from the beam collisions. This aspect is crucial to perform time-dependent measurements, core of the Belle II physics program \cite{physics_book}. The six-layer VXD, discussed in~\autoref{sec:VXD}, can determine the position of the vertices with a precision better than \SI{100}{\micro m} \cite{Belle-II:2010dht}, allowing to reconstruct secondary vertices, i.e. the decay position of the particles coming from B decays, and also from $\tau$ leptons and D mesons.

The event kinematics is illustrated in~\autoref{fig:decay_vertex}. The two B mesons are produced in an entangled quantum state, so when from the decay products of one of the two it is possible to assign its flavor (for example $B^{0}$, identified as $B_{tag}^{0}$) accordingly can be assigned that of the other, which will be the opposite ($\bar{B}^{0}$, called $\bar{B}_{phys}^{0}$).

\begin{figure}[h!]
\centering
\includegraphics[scale=.9]{vertex_decay_1}
\caption{Example of the kinematics of the golden channel of Belle II experiment.}
\label{fig:decay_vertex}
\end{figure}

After this reconstruction, both B decay vertex positions in the longitudinal direction $\textit{z}_{1}$ and $\textit{z}_{2}$ are evaluated, in order to compute their difference:

\begin{equation}
\Delta \textit{z} = \textit{z}_{1} - \textit{z}_{2} = (\beta\gamma)_{\Upsilon(4S)}c\Delta t
\end{equation}

where $\Delta t$ is the proper time decay difference. \\
Therefore this topology allows to transform a temporal information in a spatial one that we are able to measure using a high precision vertex detector. Without the boosted center of mass none of it could be possible, and this is an essential feature for an asymmetric B-factory. 


%---------------------------------------------
%			1.2
%---------------------------------------------
\section{SuperKEKB accelerator}

Belle II sensitivity in the precision measurements is feasible especially thanks to the extraordinary performance of the SuperKEKB accelerator which host the (almost) hermetic detector. This complex facility is the result of efforts and efficient collaboration between the researches of KEK laboratory and all the international working groups that participate to the experiment.


\subsection{The facility}

SuperKEKB\cite{Ohnishi:2013fma, SuperB:2013cxb} (\autoref{fig:superkekb}) is an asymmetric $e^{+}e^{-}$ collider with a circumference of \SI{3}{km} and a center-of-mass energy peak equal to  $\sqrt{s}$ = \SI{10.58}{GeV}, which corresponds to the mass of the $\Upsilon(4S)$ resonance.
Compared to its predecessor KEKB (which started its operation in 1998 and concluded it in 2010\cite{Belle:2012iwr}, reaching a peak luminosity of \SI{2.11e34}{cm^{-2} s^{-1}}), the current accelerator has a target luminosity of \SI{6e35}{cm^{-2} s^{-1}}, and already achieved a record luminosity of \SI{4.7e34}{cm^{-2} s^{-1}} in July 2022 \cite{Ohnishi:2023tds}. This target is possible using a new scheme to accelerate and collide the beams, the so called \textit{nano-beam scheme} (\autoref{sec:nano_beam}). 


\begin{figure}[h!]
\centering
\includegraphics[scale=0.8]{SuperKEKB2}
\caption{SuperKEKB accelerator structure.}
\label{fig:superkekb}
\end{figure}


\bigskip

\textbf{Luminosity}\\

Instantaneous luminosity is one of the key parameters of any accelerator and it represents the interaction rate per unit of cross section between colliding particles (\autoref{eq:lum}). Inverting this equation is possible to obtain N, namely the number of the physical events produced in the interaction with a given luminosity:

\begin{equation}
L =\frac{1}{\sigma}\frac{dN}{dt}  \qquad   \Rightarrow \qquad  N = \int_{0}^{T} L\sigma dt
\label{eq:lum}
\end{equation}

where T is the duration of the experiment and $\sigma$ the cross section of the physical process of interest. Luminosity is dependent from both machine and beam parameters. With respect to this, it can be expressed as:

\begin{equation}
L = \frac{N_{-} N_{+}}{2\pi\sqrt{\sigma_{x_{-}}^{2} + \sigma_{x_{+}}^{2}} \sqrt{\sigma_{y_{-}}^{2} + \sigma_{y_{+}}^{2}}} n_{b}f_{rev}R
\end{equation}

where ''$\pm$'' denotes positrons and electrons beam respectively, $\sigma_{x,y_{\pm}}$ represents the horizontal and vertical beam size, $N_{+,-}$  is the number of particles in a bunch, $n_{b}$ the number of bunches, $f_{rev}$ the revolution frequency, and R the geometrical loss factor. In the interaction region, where the beam-beam interaction becomes important, the following formula is commonly used:

\begin{equation} \label{eq:luminosity_eq}
L = \frac{\gamma_{\pm}}{2er_{e}} \bigg(1 + \frac{\sigma_{y}^{*}}{\sigma_{x}^{*}} \bigg) \bigg(\frac{I_{\pm}\xi_{y\pm}}{\beta^{*}_{y}} \bigg) \bigg(\frac{R_{L}}{R_{\xi_{y\pm}}} \bigg)
\end{equation}

where the starred parameters refer to their value at the Interaction Point (IP).  $I$ is the beam current, $\beta_{y}^{*}$ the vertical beta function at the IP, $\xi_{y\pm}$ is the vertical beam parameter which includes the horizontal beta function at the IP, the horizontal emittance, the bunch length and the crossing angle between the beams. $R_{L}$ and $R_{\xi_{y\pm}}$ are the reduction factors due to geometrical loss such as the hourglass effect and finite crossing.\\

As already mentioned, SuperKEKB holds the current world record in luminosity (with $\beta^{*}_{y}$= \SI{1.0}{mm}) and in the future the target will be to reach \SI{6e35}{cm^{-2} s^{-1}} (by the 2030s), by increasing currents beam and reducing their size at the IP, through the decrease of the betatron function down to $\beta^{*}_{y}$= \SI{0.3}{mm}. 
However, the increase in charge in the bunch, causes a reduction of the Touschek lifetime and the injection system is unable to compensate for the loss \cite{Iida:2023soe}.
This process makes also the beam-induced background increase significantly, risking deterioration and poor functioning of the detector.
It has been estimated that the background should remain acceptable up to a luminosity value equal to \SI{2.8e35}{cm^{-2} s^{-1}} with $\beta^{*}_{y}$= \SI{0.6}{mm} \cite{Natochii:2022vcs}. Thus, the possibility to achieve higher luminosity is closely related to an upgrade plan of both the detector and the accelerator.


\subsection{Nano-beam scheme} \label{sec:nano_beam}

We have seen that the beta function at the IP ($\beta^{*}$) is a decisive factor to define the luminosity. To be able to ramp the luminosity up, it is necessary to reduce the value of $\beta$ depending also, but not only, on the variation of the other machine parameters that appear in the~\autoref{eq:luminosity_eq}.


This new scheme, originally designed by P. Raimondi \cite{SuperB:2013cxb}, dictates that the beam bunches have to collide with sufficiently small $\sigma_{x}^{*}$ and at large angle. In case of SuperKEKB the angle is equal to \SI{83}{mrad} at the IP (larger with respect to the crossing angle used in KEKB) with the beam size of \SI{50}{nm} in the vertical direction and \SI{10}{\micro m} in the horizontal direction (in~\autoref{fig:beam_scheme_comparison} a simplified representation of the differences).

This strategy also helps to reduce the \textit{hourglass effect} \cite{Ohnishi:2013fma}, which happens when the $\beta^{*}$ is comparable or smaller than the bunch length, causing a decrease in luminosity. With a large crossing angle, the overlap length, which is the effective bunch length, is much shorter than the bunch length along the beam axis. \\


\begin{figure}[h!]
\centering
\includegraphics[scale=0.6]{nano_beam_scheme0}
\caption{Comparison between the beam schemes used in KEKB and SuperKEKB.}
\label{fig:beam_scheme_comparison}
\end{figure}

Using a crossing angle large enough has other positive implications on the operation of the accelerator and its further improvements, including allowing the placement of a new focusing system at the IP (which may require more space), considering a future redesign of the interaction region.


%---------------------------------------------
%			1.3
%---------------------------------------------
\section{Belle II detector}


The Belle II detector is a general-purpose spectrometer which consists of a concentric sub-detectors sequence placed around the beryllium beam pipe of \SI{10}{mm} radius, around the IP of the two beams. Here we will go trough a brief description of the sub-detectors (\autoref{fig:belle_detector}) going in order from the beam pipe outwards: the Vertex Detector, the Central Drift Chamber, the TOP and the ARICH, the Electromagnetic Calorimeter and the $K_{L}$ and Muon detector \cite{physics_book, Belle-II:2010dht, Adachi:2018qme}.\\

\begin{figure}[h!]
\centering
\includegraphics[scale=.8]{belle_detector}
\caption{Belle II detector.}
\label{fig:belle_detector}
\end{figure}

In~\autoref{tab:detector_summary} a summary of the main characteristics of all sub-detectors.

\begin{sidewaystable*}
\centering
\caption{Summary of the detector components\cite{physics_book}.}
\label{tab:detector_summary}
\small
\begin{tabularx}{1.0\linewidth}{lcXXXX} 
\hline
Purpose & Name & Component & Configuration & Readout channels & $\theta$ coverage \\%& Performance\\
\hline
Beam pipe & Beryllium& & Cylindrical, inner radius 10 mm, 10 $\mu$m Au, 0.6 mm Be,
1 mm paraffin, 0.4 mm Be \\
\hline
Tracking 	& PXD& Silicon Pixel (DEPFET)& Sensor size: 15$\times$(L1 136, L2 170) mm$^2$, Pixel size: 50$\times$(L1a 50, L1b 60, L2a 75, L2b 85) $\mu$m$^2$; two layers at radii: 14, 22 mm & 10M & [17$^\circ$;150$^\circ$]\\ 
 	   	& SVD& Silicon Strip& Rectangular and trapezoidal, strip pitch: 50(p)/160(n) - 75(p)/240(n) $\mu$m, with one floating intermediate strip; four layers at radii: 38, 80, 115, 140 mm  & 245k& [17$^\circ$;150$^\circ$]\\  
		& CDC& Drift Chamber with He-C$_2$H$_6$ gas& 14336 wires in 56 layers, inner radius of 160mm outer radius of 1130 mm & 14k & [17$^\circ$;150$^\circ$]\\ 
\hline
Particle ID & TOP& RICH with quartz radiator &  16 segments in $\phi$ at $r\sim120$ cm, 275 cm long, 2cm thick quartz bars with 4$\times$4 channel MCP PMTs& 8k& [31$^\circ$;128$^\circ$] \\
				& ARICH& RICH with aerogel radiator & $2\times$2 cm thick focusing radiators with different $n$, HAPD photodetectors & 78k & [14$^\circ$;30$^\circ$]\\
\hline
Calorimetry& ECL&  CsI(Tl)& Barrel: $r=125-162$cm, end-cap: $z=-102-+196$cm & 6624~(Barrel), 1152~(FWD), 960~(BWD) & [12.4$^\circ$;31.4$^\circ$], [32.2$^\circ$;128.7$^\circ$], [130.7$^\circ$;155.1$^\circ$]\\% & $\frac{\sigma E}{E}=\frac{0.2\%}{E}\oplus\frac{1.6\%}{\sqrt[4]{E}}\oplus1.2\%\sim 1.7\%$\\
\hline
Muon ID & KLM& barrel:RPCs and scintillator strips 			& 2 layers with scintillator strips  and 12 layers with 2 RPCs& $\theta$ 16k, $\phi$ 16k & [40$^\circ$;129$^\circ$] \\
			& KLM& end-cap: scintillator strips 	& 12 layers of (7-10)$\times$40 mm$^2$ strips & 17k & [25$^\circ$;40$^\circ$], [129$^\circ$;155$^\circ$]\\		
%\hline
%Trigger \\
\hline
\end{tabularx}
\end{sidewaystable*}



\subsection{Vertex Detector (VXD)}\label{sec:VXD}


The \textbf{VerteX Detector (VXD)} is composed by two devices divided into layers, the silicon Pixel Detector (PXD) and the Silicon Vertex Detector (SVD), for a total of six layers around the beam pipe.\\
The two inner layers of PXD (L12) consist of pixelated sensors based on the depleted field effect transistor (DEPFET) technology, realised with very thin (< \SI{100}{\micro m}) sensors which minimise multiple scattering, thus improving the tracking resolution for low-momentum particles. They are at a radius of \SI{14}{mm} and \SI{22}{mm}, respectively. \\
The remaining four layers of SVD (L3456) instead, are equipped with double-sided silicon strip (DSSD) sensors (at \SI{39}{mm}, \SI{80}{mm}, \SI{104}{mm} and \SI{135}{mm} respectively). Since a lower background rate is expected with respect to PXD, DSSD provides similar performance with a much smaller number of readout channels.
These layers are mainly used for tracking/vertexing and also for particle identification (PID), through the measurement of the energy loss (dE/dx).\\

We can notice in~\autoref{fig:VXD} that because of the essential asymmetric configuration of the beam energies and the consequent boost of the particles produced in the collisions (\autoref{sec:vertex_decay}), the structure of the vertex detector is asymmetric along the longitudinal axis.

\begin{figure}[h!]
\centering
\subfigure{\includegraphics[scale=0.9]{VXD1}}\quad
\subfigure{\includegraphics[scale=0.8]{VXD2}}\\
\caption{A schematic view of the Belle II vertex detector with a Be beam pipe and the six layers of PXD and SVD.}
\label{fig:VXD}
\end{figure}


\subsection{Central Drift Chamber (CDC)}

This is the central tracking device, with a large-volume drift chamber and small drift cells. The chamber gas is a \ch{He-C_{2}H_{6}} (50:50) mixture with an average drift velocity of \SI{3.3}{\centi\meter.\micro\second^{-1}} and a maximum drift time of about \SI{350}{\nano s} for a \SI{17}{mm} cell size.\\
The CDC contains 14336 wires arranged in 56 layers either in \emph{axial}  (aligned with the solenoidal magnetic field) or \emph{stereo} (skewed with respect to the axial wires) orientation (\autoref{fig:CDC}). 
Combining the information from both the axial and the stereo layers it is possible to reconstruct full three-dimensional helix charged tracks and measure their momenta.
It also provides information for PID by measuring the ionization energy loss, which is particularly useful for low-momentum particles.

\begin{figure}[h!]
\centering
\includegraphics[scale=.9]{CDC}
\caption{Schematic view of the CDC drift cells: blue dots represent the axial wires and the pink empty ones the stereo wires.}
\label{fig:CDC}
\end{figure}


\subsection{Particle identification system (TOP e ARICH)}

The \textbf{TOP (Time Of Propagation)} is a special kind of Cherenkov detector used for PID in the barrel region. It employs the two-dimensional information of a Cherenkov ring image, obtained from the time of arrival and the impact position of Cherenkov photons at the photodetector at one end of a \SI{2.6}{m} quartz bar. It is composed by 16 detector modules, each one consisted of a \numproduct{45 x 2}~\unit{cm} quartz bar (Cherenkov radiator) with a small expansion volume (about \SI{10}{cm} long) at the sensor end of the bar (\autoref{fig:TOP}). 

In order to achieve a single-photon time resolution of about \SI{100}{ps} (required for a good PID), 16-channel microchannel plate photomultiplier tubes (MCP-PMT) are employed, specifically developed for this purpose.\\

\begin{figure}[h!]
\centering
\subfigure[A schematic view of the TOP radiator.]{\includegraphics[scale=0.6]{TOP_quartz_radiator}}\\
\subfigure[A side view of the TOP radiator.]{\includegraphics[scale=0.6]{TOP_side_view}}\\
\caption{TOP detector.}
\label{fig:TOP}
\end{figure}

The \textbf{ARICH (Aerogel Ring Imaging CHerenkov)} completes the coverage of the PID system in the forward endcap region. It is a proximity focusing Cherenkov ring-imaging detector which adopts aerogel as Cherenkov radiator. In particular this detector employs a novel method to increase the number of detected Cherenkov photons: two \SI{2}{cm}-thick layers of aerogel with slightly different refractive indices ($n_{1}$ = 1.045 upstream, $n_{2}$= 1.055 downstream) that increase the photon yield without degrading the Cherenkov angle resolution (\autoref{fig:ARICH}).

Hybrid avalanche photon detectors (HAPD) are exploited as single-photon-sensitive high-granularity sensors. Here photo-electrons are accelerated over a potential difference of about \SI{8}{\kilo V} and are detected in avalanches photodiodes (APD).\\

\begin{figure}[h!]
\centering
\subfigure{\includegraphics[scale=0.5]{ARICH1}}\quad
\subfigure{\includegraphics[scale=0.85]{ARICH2}}\\
\caption{ARICH detector.}
\label{fig:ARICH}
\end{figure}

The main task of these detectors is to improve the K/$\pi$ separation from about $\approx$ \SI{1}{GeV/c} until 3.5 and \SI{4}{GeV/c} of momentum, respectively.

\subsection{Electromagnetic calorimeter (ECL)}

The \textbf{ECL} is a highly segmented array of thallium-doped caesium iodide CsI(Tl) crystals assembled in a \SI{3}{m} long barrel section with a radius of \SI{1.25}{m}, and two endcap discs located at \SI{2}{m} (forward) and \SI{1}{m} (backward). All of them are instrumented with a total of 8736 crystals, covering about 90\% of the solid angle in the center-of-mass system. 

This detector is used to detect gamma rays and to identify electrons separating them from hadrons, especially pions.

\subsection{$K_{L}$ Muon detector (KLM)}

The $K_{L}$ and Muon detector (KLM) consists of an alternating sandwich of \SI{4.7}{cm}-thick iron plates and active detector elements located outside the superconducting solenoid that provides a \SI{1.5}{T} magnetic field. The iron plates serve as the magnetic flux return joke for the solenoid as well as absorber for hadrons. They provide 3.9 interaction lengths or more of material, beyond the 0.8 interaction lengths of the calorimeter in which $K_{L}^{0}$ mesons can shower hadronically. The active detector elements have been chosen in order to cope with the reduction of the detector efficiency under the SuperKEKB background rates: resistive plate chambers (RPCs) originally installed for the KEKB/Belle experiment for the outermost active layers, while in the two innermost layers of the barrel and endcap regions, scintillator strips with wavelength-shifting fibers are used, readout by silicon photomultipliers (SiPMs).\\

\subsection{Trigger system}

The trigger system of Belle II has the role to identify events of interest during data-taking at SuperKEKB, where high background rates are expected. 
This system is divided into two levels: a hardware-based low-level trigger (L1) and a software-based high-level trigger (HLT), implemented in the data acquisition (DAQ) system. 

\begin{itemize}
\item \textbf{L1}: based mainly on fast track reconstruction in the CDC and on ECL energy, has a latency of \SI{5}{\micro s} and a maximum trigger output rate of \SI{30}{kHz}, limited by the read-in rate of the DAQ.
\item \textbf{HLT}: is a key component of the DAQ, used to fully reconstruct events that pass the L1 trigger selection. It has to reduce online event rates to \SI{10}{kHz} for offline storage and it must identify track regions of interest for PXD readout in order to reduce data flux. It fully reconstructs events with offline reconstruction algorithms, using all detectors information except for the PXD.
\end{itemize}



%---------------------------------------------
%			1.4
%---------------------------------------------
\section{Current state of data taking} \label{sec:perspectives}

SuperKEKB accelerator reached a peak luminosity of \SI{4.7e34}{cm^{-2}s^{-1}} and Belle II accumulated almost \SI{428}{\femto b^{-1}} before the beginning of Long Shutdown 1 (LS1) in July 2022. (\autoref{fig:total_luminosity}).

\begin{figure}[h!]
\centering
\includegraphics[scale=.6]{daily_luminosity}
\caption{Total recorded integrated luminosity before Long Shutdown 1.}
\label{fig:total_luminosity}
\end{figure}

The target of SuperKEKB is to achieve a \textit{$L_{inst}$} = \SI{6e35}{cm^{-2}s^{-1}} and to increase the integrated luminosity from \SI{428}{\femto b^{-1}} (current value, starting in 2019) to \SI{50}{ab^{-1}} (as shown in the projection plot in~\autoref{fig:lumy_projection}).\\

\begin{figure}
\centering
\includegraphics[scale=.7]{lumi_project}
\caption{Luminosity projection plot (plan for the coming years).}
\label{fig:lumy_projection}
\end{figure}


A three-phase upgrade program has been planned \cite{Forti:2022mti}:

\begin{itemize}
\item \textbf{short term}: year 2023. Long Shutdown 1 (LS1) started in July 2022, and will be concluded at the end of 2023. The main purpose of the shutdown was the installation of a complete PXD, since in the 2019 installation only two ladders were included in layer 2. In addition, significant maintenance and improvement work has been carried out both for Belle II and SuperKEKB.
\item \textbf{medium term}: approximately year 2028-29. Long Shutdown 2 (LS2) will probably be needed for the upgrade of the Interaction Region (IR) to reach a new luminosity target $\textit{L}_{peak}$ = \SI{6e35}{cm^{-2}s^{-1}}. 
Several open questions and difficulties have triggered many studies and discussions about a possible redesign of the machine lattice during this phase. In particular it would be necessary to deal with the limitation of the optics of the machine, concerning the further increasing of the luminosity and accordingly of the backgrounds rates. A new Vertex Detector might be also required, to accommodate the new IR design, and other sub-detectors upgrades are possible. 
\item \textbf{long term}: years > 2032. Studies have started to explore upgrades beyond the currently planned program, such as beam polarization and ultra-high luminosity and so possibly $\textit{L}_{peak}$ in excess of \SI{1e36}{cm^{-2}s^{-1}}. While the beam polarization has a concrete proposal \cite{USBelleIIGroup:2022qro, Liptak:2023rvu}, for ultra-high luminosity studies have just started.
\end{itemize}

At time of writing LS1 and the installation of the complete pixel detector (PXD) has just been completed. The restart of data taking is planned at the beginning of 2024.

