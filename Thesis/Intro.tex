% --------------------------------------------
%		INTRODUCTION
%---------------------------------------------


\chapter*{Introduction}
\addcontentsline{toc}{chapter}{Introduction}

Belle II is a particle physics experiment located at KEK laboratory in Tsukuba (Japan), 100 km away from Tokyo. The detector is a general-purpose spectrometer and it is placed along the SuperKEKB accelerator, a second generation flavor-factory which operates at the luminosity frontier, holding the world record of istantaneous luminosity with $L_{ist}$ = \num{4.7e34} $cm^{-2} s^{-1}$. 

SuperKEKB is the upgrade of the preceding facility KEKB (operational from 1998 to 2016) and it consists in a 3 km-circumference asymmetric accelerator which collides electrons and positrons beams (with energy of 7 GeV and 4 GeV, respectively) at a center-of-mass energy near the $\Upsilon$(4S) resonance ($\sqrt{s}$ = 10.58 GeV). It started its data taking in March 2019.

In the first years of the next decade, the collider aims to collect  an unrivaled dataset of 50 $ab^{-1}$ (x50 Belle dataset, x100 BaBar dataset) and to reach a new peak of istantaneous luminosity, in order to study the charge-parity violation in B mesons system with more precision and to search for new hints of pyhsics beyond the Standard Model.

To achieve these challenging targets, it would be necessary a significant upgrade of the accelerator and its main components (in particular nearby the Interaction Region) and also of the detector. As a matter of fact, to the increase in luminosity corresponds not only large data collected and greater possibility to study rare processes, but also higher doses of radiation and greater level of backgrounds which could undermine the integrity and the operation of the Belle II detector.
In particular the subdetectors which are closest to the beam pipe are those more exposed to severe conditions, like the vertex detector (VXD), composed by the inner pixel detector (PXD, made by layers of pixels) and the outermost silicon vertex detector (SVD, made by layers of strips). They deal with the reconstruction of charged particle tracks and of decay vertices with high performance. Recent studies have shown that the current detector could operate efficiently up to a luminosity of $L_{ist}$ = \num{2e35} $cm^{-2} s^{-1}$, but safety margins are not so large. 
It is precisely in this context that different upgrade projects have been proposed, which intend to strengthen new detectors, for the purposes of make them more resistant even in harsher working conditions, while the luminosity will be gradualy increased. 

This thesis focuses especially on the VerTeX Detector (VTX) proposal (the one chosen for the final upgrade), whose program provides for replacing the whole VXD with fully-pixelated five layers at different distance from the beam pipe, equipped with the same tipe of sensor technology, which is the CMOS Depleted Monolihic Active Pixel Sensors (DMAPS). 

The good results achieved by the ALICE experiment (LHC), which employed the same technology, have pushed for this solution which has proven to be reliable and promising in maintaining low occupancy in substantial level of backgrounds and good radiation hardness even after irradiation. 

In order to fullfil the physics requirements of Belle II experiments, a new silicon sensor is being designed, called OBELIX, exploiting the 180 nm TowerJazz Semiconductor process. Developments will ensure a faster, lighter and highly granular chip, reducing the material budget and as consequence improving tracks and vertices reconstruction despite the worse expected background environment. 

OBELIX planning is based on studies done on previous prototypes, among which TJ-Monopix 2, whose characterization is the main topic of this work. Continuous laboratory experiments and beam tests have been conducted and are still in progress, in order to study the efficiency of the chip before and after irradiation, its power consumption and also to fully characterize its electrical features. 
In particular, the analysis presented in this thesis, has completely characterized the response of the pixel matrix, returning important results that have allowed to interpret data obtained from measurements taken during the Test Beam at Desy in July 2022 and that are being used in the design of the OBELIX chip.


\autoref{ch:BelleII} briefly introduces some of the open questions in the Standard Model, in order to depict the background of the Belle II physics program. Then a short description of the SuperKEKB accelerator and Belle II detector is given, too. 

\autoref{ch:upgrade} presents the fundamental reasons behind the choice of an upgrade. The primary sources of the experiment background are described in a few words, in order to understand the limitation of the detector, for increasingly higher luminosity values. Eventually a summary of the four main upgrade proposals for the vertex detector is presented, which are distinguished by the different type of sensors employed: Depleted Field Effect Transistor (DEPFET), Thin and Fine-Pitch SVD, Silicon On Insulator (SOI) and CMOS Monolithic Active Pixels Sensors.


\autoref{ch:VTX} examines in depth the VerTeX detector (VTX) upgrade program, which involves the CMOS Monolithic Active Pixel Sensors as fundamental components of the five layers of the final vertex detector. Continuous studies and simulations are ongoing to test the performance, and some of them are shown here. Moreover the specifications and the implementation of the new chip thought for this proposal (OBELIX) are described. The innovative sensor has to fulfill the requirements of Belle II experiment, even in extreme environment due to higher doses of radiation and backgrounds.

\autoref{ch:CMOS} describes the principles underlying the operation of semiconductor detectors and some different type of sensors which use this technology, like the hybrid and monolithic pixel sensors. In particular the CMOS Monolithic Active Pixel Sensors technology is presented, on which the entire developments of the OBELIX chip is founded. In the end the history of the developments that led to the TJ-Monopix chip series is retraced, in order to better understand the main features of the last one, TJ-Monopix 2, whose characterization is the work of this thesis.

\autoref{ch:TJ2} lastly shows the results obtained from laboratory measurements and tests conducted on the TJ-Monopix 2 chip. The threshold distributions for all the different types of front-end circuits implemented in the matrix have been studied, together with their dispersion and noise. The absolute calibration of the whole matrix have been done, studying hhe Time Over Threshold curves (which is a time width signal processing method used in this prototype) and employing also different radioactive sources available in the laboratory. Additionally different register settings have been examined in the interest of operating the matrix at lower threshold, that is crucial to keep high efficiency even after irradiation. During this investigation, a cross-talk issue has been discovered and therefore studied to understand its causes and possible solutions.





