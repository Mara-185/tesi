% --------------------------------------------
%		INTRODUCTION
%---------------------------------------------


\chapter*{Introduction}
\addcontentsline{toc}{chapter}{Introduction}

Belle II is a particle physics experiment located at the KEK laboratory in Tsukuba (Japan). The detector is a general-purpose spectrometer to study electron-positron collisions produced by the SuperKEKB accelerator, a second generation flavor-factory which operates at the luminosity frontier, holding the world record of instantaneous luminosity with $L_{peak}$ = \SI{4.7e34}{cm^{-2}.s^{-1}}.

SuperKEKB is the upgrade of the preceding facility KEKB (operational from 1998 to 2010) and it consists in a 3 km-circumference asymmetric accelerator which collides electrons and positrons beams at a center-of-mass energy near the $\Upsilon$(4S) resonance ($\sqrt{s}$ = \SI{10.58}{GeV}). It started its data taking in March 2019.

In the next decade, the collider aims to collect  an unrivaled dataset of \SI{50}{ab^{-1}} (x50 Belle dataset, x100 BaBar dataset) and to reach a peak luminosity of \SI{6e35}{cm^{-2}.s^{-1}}. This will allow to study the charge-parity violation in B mesons system with more precision and to search for new hints of physics beyond the Standard Model.

To achieve these challenging targets, it will be necessary a significant upgrade of the accelerator and its main components (like the injection system and the equipment nearby the interaction region), probably requiring the installation of a new vertex detector. As a matter of fact, to the increase in luminosity corresponds not only large data collected and greater possibility to study rare processes, but also higher doses of radiation and larger backgrounds, which could undermine the integrity and the operation of the Belle II detector.
In particular the subdetectors which are closest to the beam pipe are those more exposed to severe conditions, like the vertex detector (VXD), composed of the inner pixel detector (PXD, made of layers of pixels) and the outermost silicon vertex detector (SVD, made of layers of strips). They allow the reconstruction of charged particle tracks and of decay vertices with high performance. Recent studies have shown that the current detector could operate efficiently up to a luminosity of $L_{inst}$ = \SI{2e35}{cm^{-2}.s^{-1}}, but safety margins are not large. 
Consequently, in this context, different upgrade projects have been proposed, which intend to design a new vertex detector, making it more resistant even in harsher working conditions, while the luminosity will be gradually increased. 

This work focuses especially on the VerTeX Detector (VTX) proposal (the one chosen for the upgrade), replacing the whole VXD with five layers of fully pixelated sensors based on the CMOS Depleted Monolithic Active Pixel Sensor (DMAPS) technology.

The good results achieved by the ALICE experiment (LHC, CERN), which employed the same technology, have suggested this solution which has proven to be reliable. The current developments, aimed at making the chips much faster than in ALICE, are promising in maintaining low occupancy, despite the worse expected background environment, and good performance even after irradiation. 

In order to fulfill the physics requirements of Belle II experiment, a new silicon sensor is being designed, called OBELIX, fabricated using the \SI{180}{nm} TowerJazz Semiconductor process. Developments will ensure a faster, lighter and highly granular chip, reducing the material budget and improving tracks and vertices reconstruction. 

OBELIX planning is based on studies done on previous prototypes, among which TJ-Monopix 2, whose characterization is the main topic of this thesis. Laboratory and beam tests have been conducted and are still in progress, in order to study the efficiency of the chip before and after irradiation, its power consumption, and to fully characterize its electrical characteristics. 
In particular, we have characterized the response of the pixel matrix, extracting important results that have allowed to interpret data taken during the Test Beam at Desy (June 2022), and that are being used in the design of the OBELIX chip. 
In more detail, the threshold distributions for all the different types of front-end circuits implemented in the matrix have been studied, together with their dispersion and noise distributions. 
The calibration of the Time Over Threshold curves (which is a time width signal processing method used in this prototype) has been done by internal injection tests. The absolute calibration of the whole matrix has been achieved, employing a \ch{^{55}Fe} radioactive source. Other radioactive sources have been used too, in order to check the trend of the TOT curves for charge values not accessible by internal injection. 
Additionally, different register settings have been examined in the interest of operating the matrix at low threshold, that is crucial to keep high efficiency even after irradiation. For this reason, several tests have been conducted to tune the threshold, in order to reduce the dispersion and make the threshold on the matrix as uniform as possible.
During this investigation, a cross-talk issue has been discovered and therefore studied to understand its causes and possible solutions to mitigate this effect.

The measurements performed in this thesis have been used for the design of OBELIX chip, that should be submitted for fabrication in the next few months.



\autoref{ch:BelleII} briefly introduces some of the open questions in the Standard Model, in order to depict the background of the Belle II physics program. A short description of the SuperKEKB accelerator and Belle II detector is also given. 

\autoref{ch:upgrade} presents the main arguments supporting the Belle II upgrade program. The primary sources of the experiment background are summarized, to understand the limitation of the detector and the accelerator, for increasingly higher luminosity values. Eventually a summary of the four main upgrade proposals for the vertex detector is presented, which are distinguished by the different type of sensors employed: Depleted Field Effect Transistor (DEPFET), Thin and Fine-Pitch strip detectors, Silicon On Insulator (SOI) and CMOS Monolithic Active Pixels Sensors.

\autoref{ch:CMOS} describes the principles underlying the operation of semiconductor detectors and some different type of sensors which use this technology, like the hybrid and monolithic pixel sensors. In particular the CMOS Monolithic Active Pixel Sensors technology is presented, on which the entire developments of the OBELIX chip is based. In the end, the history of the developments that led to the TJ-Monopix chip series is retraced, in order to better understand the main features of the last one, TJ-Monopix 2, which represents the starting point for OBELIX design, and whose characterization is the work of this thesis.

\autoref{ch:VTX} examines in depth the VerTeX detector (VTX) upgrade program, which involves the CMOS Monolithic Active Pixel Sensors as fundamental components of the five layers of the final vertex detector. Studies and simulations are ongoing to test the performance, and some of them are shown here. The specifications and the implementation of the new chip (OBELIX) under design for this proposal are described. The innovative sensor has to fulfill the requirements of Belle II experiment, even in extreme environment due to higher doses of radiation and backgrounds.


\autoref{ch:TJ2} lastly shows the results obtained from laboratory measurements and tests conducted on the TJ-Monopix 2 chip. The response of the matrix has been studied in different working conditions, in order to analyze the behaviour at high and low threshold. The absolute calibration of the all front-end circuits implemented in the chip, has been done too.
Moreover a cross-talk issue have been discovered and analyzed, in order to understand its causes and a possible mitigation of this effect since it prevented from using the matrix at low threshold.
